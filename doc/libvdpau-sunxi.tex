\documentclass[11pt, a4paper,ngerman]{article}
\usepackage{basic}

\title{ Instructions to use \\ libvdpau-sunxi with VDR \\ on \\ \cubiebig \\ Cubieboard 1-3 \\ and other A10/A20}
\author{ ... }
\date{Compiled on \today\ at \currenttime}
\begin{document}


%============================================================= start   ==========================================================================
\maketitle
\newpage % <<------------------------------------------------------------------------------------- Newpage
\tableofcontents 
\newpage % <<------------------------------------------------------------------------------------- Newpage
%  =================================================================================================================

%\changefont{ppl}{b}{n}
%\changefont{phv}{m}{n}
%\changefont{cmss}{m}{n}
%\changefont{phv}{m}{n}
\changefont{pag}{m}{n}

\section{Hardware}
\subsection{Technical}
\begin{Verbatim}[frame=lines,
       framerule=0.2mm,framesep=3mm,
       rulecolor=\color{monoorange},
       fillcolor=\color{monogreen},
       label=\cubiebig,labelposition=topline]
  Name    :   Cubieboard 2
  Size    :   10 cm x 6 cm
  CPU     :   Allwinner A20 SoC (2 ARM-Cortex A7-Cores with 1 GHz)
  GPU     :   Mali-400MP2 (OpenGL ES 2.0/1.1)
  VPU     :   CedarX (max 2160p (Ultra HD))
  RAM     :   512MB (Test) / 1GB (Produktion) DDR3
  CONN    :   2x USB Host, 1x USB On-the-go, 1x CIR, 1x SATA
  VID-OUT :   HDMI @ 1080p
  AUD-OUT :   S/PDIF, Headphone, HDMI-Audio
  AUD-IN  :   Mikrophone, Line-In
  Storage :   4 GB NAND-Flash, 1x MicroSD
  Network :   10/100-Ethernet
  DB-Con. :   96 Pin incl I²C, SPI, LVDS
\end{Verbatim}

\section{Software}
\subsection{General}

To use CedarX with VDR and Softhd-device the standard 80MB Video-Memory is not enough!\\

The Bad $\rightarrow$ it is hardcoded in the Kernel since 3.4.7x.\\

There are two way's to fix it (on both you have to recompile the kernel):\\
$\bullet$ change the value in the kernel. \\
$\bullet$ disable CMA and use the kernelparameters.

\subsection{Installation Driver and Library's}

\subsubsection{xf86 Driver}

\begin{mintedbox}[breakable=true,
%ribbonbox,
 bottomrule=0.5mm,
 width=\paperwidth-3cm,
 boxsep=1mm, 
 enhanced=true,
 colframe = monoblack,
 drop fuzzy shadow,
 colback = black
 ]{xf86\ Driver\ installation}%
 

     \inputminted[firstline=3,lastline=10, 
     linenos=true, framesep=2mm, mathescape, numbersep=5pt,tabsize=4,%fontsize=\footnotesize, 
]{bash}{includes/software.sh}%

\end{mintedbox}%

\subsubsection{FFmpeg (2.4.4)}

Alternative libav should work. \\

It's important to compile FFmpeg with "--enable-shared" flag \\

\begin{mintedbox}[breakable=true,
%ribbonbox,
 bottomrule=0.5mm,
 width=\paperwidth-3cm,
 boxsep=1mm, 
 enhanced=true,
 colframe = monoblack,
 drop fuzzy shadow,
 colback = black
 ]{FFmpeg\ installation\ (\ 2.4.4\ )}%
 

     \inputminted[firstline=14,lastline=20, 
     linenos=true, framesep=2mm, mathescape, numbersep=5pt,tabsize=4,%fontsize=\footnotesize, 
]{bash}{includes/software.sh}%

\end{mintedbox}%

\subsubsection{Libvdpau-sunxi}

\begin{mintedbox}[breakable=true,
%ribbonbox,
 bottomrule=0.5mm,
 width=\paperwidth-3cm,
 boxsep=1mm, 
 enhanced=true,
 colframe = monoblack,
 drop fuzzy shadow,
 colback = black
 ]{libvdpau-sunxi}%
 

     \inputminted[firstline=24,lastline=28, 
     linenos=true, framesep=2mm, mathescape, numbersep=5pt,tabsize=4,%fontsize=\footnotesize, 
]{bash}{includes/software.sh}%

\end{mintedbox}%

\subsubsection{Editing uEnv}

to use the Kernel-Parameter in uEnv the kernel should have build without CMA

\begin{mintedbox}[breakable=true,
%ribbonbox,
 bottomrule=0.5mm,
 width=\paperwidth-3cm,
 boxsep=1mm, 
 enhanced=true,
 colframe = monoblack,
 drop fuzzy shadow,
 colback = black
 ]{CMA disable}%
 

     \inputminted[firstline=1,lastline=28, 
     linenos=true, framesep=2mm, mathescape, numbersep=5pt,tabsize=4,%fontsize=\footnotesize, 
]{bash}{includes/kernel.conf}%

\end{mintedbox}%
\vspace{0.4cm}

edit the uEnv.*  \\


\begin{mintedbox}[breakable=true,
%ribbonbox,
 bottomrule=0.5mm,
 width=\paperwidth-3cm,
 boxsep=1mm, 
 enhanced=true,
 colframe = monoblack,
 drop fuzzy shadow,
 colback = black
 ]{uEnv}%
 

     \inputminted[firstline=1,lastline=28, 
     linenos=true, framesep=2mm, mathescape, numbersep=5pt,tabsize=4,%fontsize=\footnotesize, 
]{bash}{includes/uEnv.txt}%

\end{mintedbox}%
\vspace{0.4cm}

\color{monoorange}
\underline{\textbf{sunxi\_ve\_mem\_reserve=190}} \\
\color{black}
is for the reservation of the memory used by the CedarX \\


\color{monoorange}
\underline{\textbf{hdmi.audio=EDID:0}} \\
\color{black}
for HDMI-Audio \\

\color{monoorange}
\underline{\textbf{disp.screen0\_output\_mode=1920x1080p50}} \\
\color{black}
1920x1080p50 is nessesary for use with SD- or HD-TV channels. \\


\newpage % <<------------------------------------------------------------------------------------- Newpage
\subsection{Sound}

There are two way's to use the Sound-devices of the Allwinner A10/A20 SoC's\\

ALSA and Pulse-Audio \\
\subsubsection{ALSA}

To use analog audio on the HDMI-Port you have to edit the: /etc/asound.conf !\\

\begin{mintedbox}[breakable=true,
%ribbonbox,
 bottomrule=0.5mm,
 width=\paperwidth-3cm,
 boxsep=1mm, 
 enhanced=true,
 colframe = monoblack,
 drop fuzzy shadow,
 colback = black
 ]{/etc/asound.conf}%
 

     \inputminted[firstline=3,lastline=32, 
     linenos=true, framesep=2mm, mathescape, numbersep=5pt,tabsize=4,%fontsize=\footnotesize, 
]{apacheconf}{includes/asound.conf}%

\end{mintedbox}%
\vspace{0.2cm}


\subsubsection{Pulse-Audio}

for HDMI-audio: \\

\begin{mintedbox}[breakable=true,
%ribbonbox,
 bottomrule=0.5mm,
 width=\paperwidth-3cm,
 boxsep=1mm, 
 enhanced=true,
 colframe = monoblack,
 drop fuzzy shadow,
 colback = black
 ]{HDMI\ Audio\ pulseaudio}%
 

     \inputminted[firstline=3,lastline=3, 
     linenos=true, framesep=2mm, mathescape, numbersep=5pt,tabsize=4,%fontsize=\footnotesize, 
]{bash}{includes/pulseaudio.sh}%

\end{mintedbox}%
\vspace{0.2cm}

For analog-audio-jack: \\

\begin{mintedbox}[breakable=true,
%ribbonbox,
 bottomrule=0.5mm,
 width=\paperwidth-3cm,
 boxsep=1mm, 
 enhanced=true,
 colframe = monoblack,
 drop fuzzy shadow,
 colback = black
 ]{ANALOG\ Audio\ pulseaudio}%
 

     \inputminted[firstline=5,lastline=5, 
     linenos=true, framesep=2mm, mathescape, numbersep=5pt,tabsize=4,%fontsize=\footnotesize, 
]{bash}{includes/pulseaudio.sh}%

\end{mintedbox}%
\vspace{0.2cm}

\newpage % <<------------------------------------------------------------------------------------- Newpage
\subsection{Kernel Modules}

Add modules to /etc/modules. \\

\begin{mintedbox}[breakable=true,
%ribbonbox,
 bottomrule=0.5mm,
 width=\paperwidth-3cm,
 boxsep=1mm, 
 enhanced=true,
 colframe = monoblack,
 drop fuzzy shadow,
 colback = black
 ]{edit\ /etc/modules}%
 

     \inputminted[firstline=5,lastline=5, 
     linenos=true, framesep=2mm, mathescape, numbersep=5pt,tabsize=4,%fontsize=\footnotesize, 
]{bash}{includes/modules.sh}%

\end{mintedbox}%

\begin{mintedbox}[breakable=true,
%ribbonbox,
 bottomrule=0.5mm,
 width=\paperwidth-3cm,
 boxsep=1mm, 
 enhanced=true,
 colframe = monoblack,
 drop fuzzy shadow,
 colback = black
 ]{add\ following:}%
 

     \inputminted[firstline=8,lastline=25, 
     linenos=true, framesep=2mm, mathescape, numbersep=5pt,tabsize=4,%fontsize=\footnotesize, 
]{bash}{includes/modules.sh}%

\end{mintedbox}%



%
% ---------------------------------------------------------- THE VIDEO DISK RECORDER ----------------------------------------------------------
%
\newpage % <<------------------------------------------------------------------------------------- Newpage

\subsection{The Video Disk Recorder}

\subsubsection{vdr-plugin-softhddevice}

To use the OSD of the VDR you have to use the old bitmap surface of softhddevice. \\
On a normal PC with a NVIDIA-card this surfaces are not used anymore. \\

So you have to enable it again in the Makefile of the Plugin. \\

\begin{mintedbox}[breakable=true,
%ribbonbox,
 bottomrule=0.5mm,
 width=\paperwidth-3cm,
 boxsep=1mm, 
 enhanced=true,
 colframe = monoblack,
 drop fuzzy shadow,
 colback = black
 ]{Enable\ Bitmap-Surface\ in\ SoftHD-device\ Makefile}%
 

     \inputminted[firstline=32,lastline=32, 
     linenos=true, framesep=2mm, mathescape, numbersep=5pt,tabsize=4,%fontsize=\footnotesize, 
]{bash}{includes/software.sh}%

\end{mintedbox}%
\vspace{0.2cm}

And search the line with: \underline{\#\#\# Make it standard} and add \underline{-fsigned-char} at the lines: \\

\begin{mintedbox}[breakable=true,
%ribbonbox,
 bottomrule=0.5mm,
 width=\paperwidth-3cm,
 boxsep=1mm, 
 enhanced=true,
 colframe = monoblack,
 drop fuzzy shadow,
 colback = black
 ]{Add\ a\ fix\ in\ SoftHD-device\ Makefile}%
 

     \inputminted[firstline=34,lastline=37, 
     linenos=true, framesep=2mm, mathescape, numbersep=5pt,tabsize=4,%fontsize=\footnotesize, 
]{bash}{includes/software.sh}%

\end{mintedbox}%
\vspace{0.2cm}

\subsubsection{a basic VDR startscript}

\begin{mintedbox}[breakable=true,
%ribbonbox,
 bottomrule=0.5mm,
 width=\paperwidth-3cm,
 boxsep=1mm, 
 enhanced=true,
 colframe = monoblack,
 drop fuzzy shadow,
 colback = black
 ]{a\ basic\ VDR\ startscript}%
 

     \inputminted[firstline=0,lastline=35, 
     linenos=true, framesep=2mm, mathescape, numbersep=5pt,tabsize=4,%fontsize=\footnotesize, 
]{bash}{includes/vdr-startscript.sh}%

\end{mintedbox}%

\newpage % <<------------------------------------------------------------------------------------- Newpage
%==================================  OPTIONAL  =====================================================================================

\subsection{Optional}

\subsubsection{qvdpautest}

\begin{mintedbox}[breakable=true,
%ribbonbox,
 bottomrule=0.5mm,
 width=\paperwidth-3cm,
 boxsep=1mm, 
 enhanced=true,
 colframe = monoblack,
 drop fuzzy shadow,
 colback = black
 ]{qvdpautest installation}%
 

     \inputminted[firstline=3,lastline=32, 
     linenos=true, framesep=2mm, mathescape, numbersep=5pt,tabsize=4,%fontsize=\footnotesize, 
]{bash}{includes/qvdpau.sh}%

\end{mintedbox}%
\vspace{0.2cm}

\newpage % <<------------------------------------------------------------------------------------- Newpage
\subsection{Notes}
\infobox{Notes}{
$\ $\\
 \centering
vdpau-sunxi is still in development. \\
$\ $\\
$\ $\\
}{$\ $}



%=========================================================================================================================
%
%-------------END
\end{document}  

\shiftkey \\
\capslockkey \\
\ejectkey \\
\pencilkey \\
\returnkey \\
\revreturnkey \\
\cubie \\
\cubiebig \\
